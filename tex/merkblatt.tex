\documentclass[10pt,a4paper^, twocolumn]{article}
\usepackage[utf8]{inputenc}
\usepackage{amssymb}
\usepackage{amsmath}
\usepackage{hyperref}
\usepackage[ngerman]{babel}
\usepackage[noheadfoot,top=1.5cm,bottom=2.5cm,left=1.5cm,right=1.5cm]{geometry}
\usepackage{fancyhdr}
\usepackage{wasysym} % \smiley :)
\usepackage{scalefnt}
\usepackage{graphicx} % scalebox, rotatebox


%%% AND: \wedge; OR: \vee
%%% neue befehle -- bitte benutzen ;-)

\newcommand{\basis}{\mathfrak} % basis in frakturschrift
\newcommand{\menge}{\mathbb} % menge in mengenschrift ^^
\newcommand{\und}{\wedge} % logisches und
\newcommand{\oder}{\vee} % logisches oder
\newcommand{\entspr}{\widehat{=}} % entspricht zeichen
\newcommand{\liminfty}[1]{\lim \limits_{#1 \rightarrow \infty}} % lim {#1} --> infty
\renewcommand{\epsilon}{\varepsilon} % *das* epsilon!
\renewcommand{\phi}{\varphi} % *das* phi!
\newcommand{\pivot}{{\fboxsep1pt \fbox{$\ast$}}} % pivot-element in stufen-matrix
\renewcommand{\d}{\mathrm{d}} % d/dx schreiben als \frac{\d}{\d x}


%%% neue befehle -- bitte benutzen ;-)
\makeatletter
\renewcommand*\env@matrix[1][*\c@MaxMatrixCols c]{%
  \hskip -\arraycolsep
  \let\@ifnextchar\new@ifnextchar
  \array{#1}}
\makeatother

\parindent 0mm % Absatzeinzug auf 0 setzen
\lhead{} % Überschrift
\rhead{} % seitenzahl ausblenden -- erst nach TOC anzeigen
\cfoot{} % seitenzahl ausblenden
\renewcommand{\headrulewidth}{0mm} % header trennlinie ausblenden

\begin{document}
\scalefont{0.8} % schriftgröße skalieren: 10pt --> 8pt

\begin{center}
  {\large Analysis Merkblatt}\\
  {}
\end{center}

\section{Zahlensysteme}
\subsection{Axiomatische Charakterisierung der Reellen Zahlen}
\subsection{Komplexe Zahlen}
\subsection{Einige nützliche Bezeichungen}
\subsection{Rechenregeln für Suprema}
\subsection{Archimedizität der Reellen Zahlen}
\subsection{Dichtheit der Rationalen Zahlen}
\subsection{Dezimaldarstellung}

\section{Ungleichungen}
\subsection{Elementare Ungleichungen}
Dreiecksungleichung: \\
$|a+b| \leq |a|+|b|$ \\
Umgekehrte Dreiecksungleichung: \\
$|a+b| \geq ||a|-|b||$ \\
Ungleichheit für geometrische Summen:\\
$1+x+x^2+x^3+\dots+x^n \leq \frac{1}{1-x} \quad (n \in \menge{N}_0, 0 \leq x < 1)$ \\
Bernoulli'sche Ungleichung:\\
$(1+x)^n \geq 1+nx \quad (n \in \menge{N}_0, x > -1)$ \\
Ungleichung zwischen geometrischem und arithmetischem Mittel: \\
$\sqrt{ab} \leq \frac{a+b}{2} \quad (a,b \geq 0)$ \\
Ungleichung vom mittleren Verhältnis:\\
	$
	\min(\frac{a_1}{b_1}, \dots, \frac{a_n}{b_n}) 
	\leq \frac{a_1+\dots+a_n}{b_1+\dots+b_n} 
	\leq \max(\frac{a_1}{b_1}, \dots, \frac{a_n}{b_n})
	\quad (b_1,\dots,b_n > 0)
	$
\subsection{Cauchy-Schwarz'sche Ungleichung}
$\sum\limits_{k=1}^n x_ky_k \leq \sqrt{\sum\limits_{k=1}^n x_k^2} \cdot \sqrt{\sum\limits_{k=1}^n y_k^2} \quad (x_1,\dots,x_n,y_1,\dots,y_n \in \menge{R})$ \\
oder auch: $\langle x,y \rangle \leq ||x|| \cdot ||y|| \quad (x,y \in \menge{R}^n)$
\subsection{Euklidische Norm}
$||x|| := \sqrt{\langle x,x \rangle} = \sqrt{\sum\limits_{k=1}^n x_k^2}$ \quad f"ur $x \in \menge{R}^n$
\subsubsection{Euklidisches Skalarprodukt}
$\langle x,y \rangle := \sum\limits_{k=1}^n x_k \cdot y_k$ \quad f"ur $x,y \in \menge{R}^n$ \\
andere Normen

\section{Folgen}
\subsection{Konvergenz von Folgen}
	Die Folge $(a_n)$ konvergiert gegen den Grenzwert $a \in \mathbb{R}$, falls für
	jede Genauigkeit $\epsilon > 0$ die Abschätzung $|a_n - a| \leq \epsilon$ für fast
	alle $n \in \mathbb{N}$ eingehalten wird.
\subsubsection{Monotonie der Grenzwertbildung}
	Es seien $a_n \to a$ und $b_n \to b$.\\
	Gilt $a_n \leq b_n \, \forall n \geq n_0 \in \mathbb{N}$, dann gilt $ a \leq b$.
\subsubsection{Einschließungsregel}
	Es sei $\forall \, n \geq n_0 \in \mathbb{N} : \, a_n \leq a_n' \leq a_n''$.
	Dann gilt 
	$$\lim_{n \to \infty} a_n = \lim_{n \to \infty} a_n'' = a 
	\qquad \Rightarrow  \qquad \lim_{n \to \infty} a_n' = a$$
\subsubsection{Konvergenz in $\mathbb{R}^d$ und $\mathbb{C}$}
	Die vektorwertige Folge\\
	$\mathbb{N} \ni n \mapsto x_n = 
		\begin{pmatrix}
			x_{n_1}\\
			\vdots \\
			x_{n_d}
		\end{pmatrix}$
	konvergiert gegen den Grenzwert $
		x = \begin{pmatrix}
			x_{1}\\
			\vdots \\
			x_{d}
		\end{pmatrix}$\\
	$$\Leftrightarrow 
		\lim_{n \to \infty} x_{n_k} = x_k 
		\quad \forall k \in \{1, \dots, d \} 
		\text{(komponentenweise Grenzwerte existieren)}$$
	Analog in $\mathbb{C}$ (isomorph zu $\mathbb{R}^2$).
\subsection{Beschränktheit konvergenter Folgen}
	Eine konvergente Folge $(a_n) \subset \mathbb{R}$ ist beschränkt.\\
	D.h. $\exists c > 0 $ mit $|a_n| \leq c \forall n \in \mathbb{N}$.\\
	Das gleiche gilt für Folgen aus  $\mathbb{R}^d$ und $\mathbb{C}$ mit $||\cdot||_2$
	bzw. dem Absolutbetrag in $\mathbb{C}$.
\subsection{Stetigkeit: Rechnen mit Grenzwerten}
	Eine Funktion $f : D \subseteq \mathbb{R}^d \to \mathbb{R}^q$ heißt 
	im Punkt $x \in D$ ihres Definitionsbereichs stetig, falls für jede Folge 
	$(x_n)$ aus $D$ gilt:
	$$
		x_n \overset{n \to \infty}{\to} x
		\qquad \Rightarrow \qquad
		f(x_n) \overset{n \to \infty}{\to} f(x)
	$$
	\subsubsection{Elementare stetige Funktionen}
	\begin{center}
	\newcommand{\infowidth}{2cm}
	\newcommand{\aufsichtwidth}{10cm}
	\begin{tabular}{|c|c|}
	\hline
	$D \subseteq \mathbb{R}$ &  $f(x) \in \mathbb{R}$ \\
	\hline
	$\mathbb{R}$ & $f(x) = c, c \in \mathbb{R}$\\
	$\mathbb{R}$ & $f(x) = |x|$\\
	$[0, \infty)$& $f(x) = \sqrt{x}$\\
	\hline
	$D \subseteq \mathbb{R}^2$ &  $f(x, y) \in \mathbb{R}$ \\
	\hline
	$\mathbb{R}^2$ & $f(x, y) = x+y$\\
	$\mathbb{R}^2$ & $f(x, y) = x \cdot y$\\
	$\mathbb{R} \times (\mathbb{R} \setminus \{0\}) $ & $f(x, y) = x \div y$\\
	\hline
	\end{tabular}
	\end{center}
	Summe, Produkt, Quotient zweier konvergenter Folgen ist konvergent. Für
	divergente/uneigentlich konvergente Folgen gilt das nicht.
	\subsubsection{Stetigkeit der Verknüpfung von Funktionen}
		Sei $f : D \subseteq \mathbb{R}^d \to \mathbb{R}^q$, 
		$g : E \subseteq \mathbb{R}^q \to \mathbb{R}^p$ mit 
		$f(D) \subseteq E$ und \\
		$g \circ f : D \subseteq \mathbb{R}^d \to \mathbb{R}^p, 
		(g \circ f)(x) := g(f(x))$ die Verkfüpfung von $f$ mit $g$. 
		Dann gilt: Wenn $f$ stetig in $x \in D$ und $g$ stetig $f(x) \in E$\\
		$\Rightarrow g \circ f$ stetig in $x \in D$
\subsection{Monotone Folgen}
\subsection{Beschränkte Folgen}
\subsubsection{Satz von Bolzano-Weierstrass}
\subsubsection{Lemma Seite 6/14, VL 6}
\subsubsection{Korollar Seite 7/14, VL 6}
\subsection{Exponentialfunktion}
\subsubsection{Abschätzungen der Exponentialfunktion}

\section{Reihen}
\subsection{Konvergenz von Reihen}
\subsubsection{Beispiele}
\subsection{Vergleichskriterien}
\subsubsection{Majorantenkriterium}
\subsubsection{Quotientenkriterium}
\subsubsection{Wurzelkriterium}

\section{Konsequenzen der Stetigkeit}
\subsection{Zwischenwertsatz} 
\subsection{Existenz von Maximum und Minimum}
\subsubsection{Lemma}
\subsubsection{Korollar}


\section{Potenzreihe}
	Sei $(a_n)_{n\in \menge{N}}$ Folge, 
	dann heißt $f:(-r,+r) \rightarrow \menge{R} \quad f(x) = \sum\limits_{k=0}^\infty a_kx^k$ Potenzreihe. \\
	$r > 0$ heißt Konvergenzradius, $r = \frac{1}{\limsup \sqrt[n]{|a_n|}}$. \\
	Welche Funktionen $f$ haben die Darstellung $f(x) = \sum\limits_{k=0}^\infty a_kx^k \quad \forall x \in (-r,r) \rightarrow$ \underline{Taylorentwicklung} \\
	$f: \menge{I} \rightarrow \menge{R}, \menge{I} = (a,b) \quad f (n+1)$-mal stetig diffbar, dann: \\
	$\forall x,a \in \menge{I}: (x) = \underbrace{ \sum\limits_{k=0}^{n} \frac{f^{(k)} (a)}{k!}(x-a)^k }_{\text{Taylorpolynom n-ten Grades von f um a}} + R_{n+1}(a,x)$ \\
	$R_{n+1}(a,x) = \begin{cases}
		\frac{1}{n!} \int\limits_a^x(x-t)^nf^{(n+1)}(t) \d t \\
		\frac{f^{(n+1)}(\xi)}{(n+1)!} (x-a)^{n+1} \quad \xi \in [x,a]
	\end{cases}$ \\
	falls f $\infty$-mal diffbar, $R_{n+1}(x) \rightarrow 0 (n \rightarrow \infty)
	\Rightarrow f(x) = \sum\limits_{k=0}^\infty \frac{f^{(k)}(a)}{k!} (x-a)^k$ \quad (Taylorreihe von f um a) \\
	$f:\menge{R} \rightarrow \menge{R}$ heißt analytisch in $x=0$, falls f als $f(x) = \sum\limits_{k=0}^\infty a_k x_k \quad \forall |x| < r$ darstellbar ist.
	\begin{itemize}
		\item $f \in C^\infty((-r,r))$
		\item $f$ analytisch, $g$ analytisch $\Rightarrow f \circ g$ analytisch mit
			$(f \circ g)(x) = \sum\limits_{k=0}^\infty \sum\limits_{j=0}^\infty (a_j \cdot b_{k-j})x^k \quad , |x| < min\{r_f, r_g\}$
		\item $a_k = \frac{f^{(k)}(0)}{k!}$
		\item $f$ analytisch, $f(0) \neq 0 \Rightarrow \frac{1}{f}$ analytisch
	\end{itemize}



\section{Kurven}
	\begin{itemize}
	\item{Def.:} Sei $\gamma : I \rightarrow \menge{R}^n$ eine diffbare Kurve, 
		dann heißt $\gamma'(t)$ Tangentialvektor (TV) oder Geschwindigkeitsvektor von $\gamma$ zum Zeitpunkt $t$ \\
		$||\gamma'(t)|| = \sqrt{\gamma'_1(t)^2 + \dots + \gamma'_n(t)^2}$ heißt Geschwindigkeit zum Zeitpunkt t \\
		$\frac{\gamma'(t)}{||\gamma'(t)||}$ heißt Tangentialeinheitsvektor (TEV) (falls $\gamma'(t) \neq 0$).
		$\gamma: I \rightarrow \menge{R}^n$ diffbar heißt regul"ar falls $\gamma'(t) \neq 0 \quad \forall t \in I$
	\item{Satz:} Eine außer an endlich vielen Stellen stetig diffbare Kurve $\gamma: I \rightarrow \menge{R}^n, I=[a,b]$ ist rektifizierbar und hat Länge
		$S(\gamma) = \int\limits_a^b || \gamma(t) || \d t$
	\item{Korollar:} Sei $f:[a,b] \rightarrow \menge{R}$ stetig diffbar, $\gamma:[a,b] \rightarrow \menge{R}^2$ \\
		$\gamma(x) = (x, f(x))$, 
		dann $S(f) = \int\limits_a^b\sqrt{1+f'^2(x)} \d x $
	\end{itemize}


\subsection{Parameterwechsel}
	Sei $\gamma : [a,b] \rightarrow \menge{R}^n$ Kurve und rektifizierbar. \\
	Ziel: $\tilde\lambda : [a', b'] \rightarrow [a,b]$ mit gleicher Spur wie $\gamma$, aber $DG = 1 \ \forall t \in [a', b']$ \\
	\begin{enumerate}
	\item{}  Sei $\phi:[a', b'] \rightarrow [a,b]$ stetig, bijektiv, dann heißt $\phi$ Parametertransformation von $[a,b]$ nach $[a',b']$ \\
	Definiere: $\tilde\gamma = \gamma \circ \phi, \tilde\gamma:[a', b'] \rightarrow \menge{R}^n$ mit $\tilde\gamma([a',b']) = \gamma(\phi([a',b'])) = \gamma([a,b])$ \\
	\item{} $S(\xi) = \int\limits_a^\xi ||\gamma'(t)|| \d t, s(0) = 0, s(b) = S(\gamma)$ \\
	$s'(\xi) = ||\gamma'(\xi)|| > 0$ \\
	$s [a,b] \rightarrow [0, S(\gamma)]$ und $s(t) = \int\limits_0^t ||\gamma'(\tau)|| \d \tau $ \\
	$s^{-1}[0,S(\gamma)] \rightarrow [a,b] \qquad s^{-1}$ wird $\phi$ sein. \\
	$\tilde\gamma'(\xi) = (\gamma \circ s^{-1})'(\xi) = \gamma'(s^{-1}(\xi)) \cdot (s^{-1})'(\xi) = 
	\gamma'(s^{-1}(\xi)\cdot \frac{1}{s'(s^{-1}(\xi))} = \gamma'(s^{-1}(\xi) \cdot \frac{1}{||\gamma'(s^{-1}(\xi))||}$
	\end{enumerate}




\section{Differentialgleichungen}
\subsection{Skalare gew"ohnliche DGL 1. Ordnung}
	lässt sich ein Anfangswertproblem schreiben als \\ $x' = f(t)g(x) \quad x(t_0) = x_0$  \\
	und sind ferner $f,g$ stetig und $g(x_0) \neq 0$, dann existiert eine lokale stetige Lösung $\phi(t)$ mit \\
	$\int\limits_{t_0}^t f(\tau) \d \tau = \int\limits_{x_0}^{\phi(t)} \frac{1}{g(\xi)} \d \xi$ \\
	Falls $g(x_o) = 0  \quad \Rightarrow \quad \phi(t) = x_0$

\subsection{Skalare lineare gew"ohnliche DGL $n$-ter Ordnung}
	$x^{(n)} + a_{n-1}x^{(n-1)} + \dots + a_0x = b(t)$ \\
	Alle Lösungen lassen sich schreiben als $\phi(t) = \phi_h(t) + \phi_s(t)$ \\
	Bestimmung der homogenen Lsg: \\
	$p(\lambda) = \lambda^n + a_{n-1}\lambda^{n-1} + \dots + a_0\lambda^0$  \\
	Seinen $\lambda_1,\dots,\lambda_s$ verschiedene Nullstellen des Polynoms $p(\lambda)$ mit den Vielfachheiten $r_1,\dots,r_s$,
	dann bilden folgende Funktionen eine Lösungsbasis $\langle \phi_h(t) \rangle$ des homogenen Problems: \\
	$e^{\lambda t}, te^{\lambda_jt},\dots,t^{r_j-1}e^{\lambda_jt} \quad ( 1 \leq j \leq s )$ \\
	Fall 1: Alle $\lambda_j$ reell $\quad \Rightarrow$ bereits reelle Lösungsbasis. \\
	Fall 2: $\lambda_k$ komplex $\Rightarrow \overline{\lambda_k}$ auch Nullstelle von $p$
		$\lambda_k = \alpha_k + i\beta_k$ und $\overline{\lambda_k}$ ersetzen durch: 
		$te^{\alpha_kt}cos(\beta_kt), t^ke^{\alpha_kt}sin{\beta_kt} \quad (1 \leq k \leq r_j-1)$

\subsection{Lineare gew"ohnliche DGL im $\menge{R}^n$}
	Sei $y = (y_1,\dots,y_n)^T \in \menge{R}^n$, dann $y' = Ay, A \in \menge{R}^{n\times n}$ Matrix \\
	Sind $y_1(t),\dots,y_n(t)$ linear unabh"angige reelle Fundamentallösungen von $y' = Ay$,
	so ist \\
	$y(t) = \alpha_1 y_1(t)+\dots+\alpha_ny_(t) \quad (\alpha_1,\dots,\alpha_n \in \menge{R}$ beliebig$)$ die allgemeine L"osung\\
	Berechnung der Fundamentall"osungen:
	\begin{enumerate}
	\item{} Sei $\lambda \in \menge{C}$ ein Eigenwert von $A$, $z \in \menge{C}^n \setminus \{0\}$ der zugeh"orige Eigenvektor,
		dann ist $y(t) = e^{\lambda t} \cdot z$ eine Lösung von $y' = Ay$  \\
		denn $y(t) = e^{\lambda t} \cdot z, Ax = Ae^{\lambda t}z = e^{\lambda t}Az = \lambda e^{\lambda t}z$ \\
		Existieren zum Eigenwert $\lambda \in \menge{C}$ linear unabh"angige Eigenvektoren $z_1,\dots,z_k$, so sind die Lösungen
		$y_1(t) = e^{\lambda t}z_1, \dots, y_k(t) = e^{\lambda t}z^k$ linear unabh"angig.
	\item{} Sei $\lambda \in \menge{C} \setminus \menge{R}$, d.h. $\lambda = \alpha + i\beta, \beta \neq 0 \quad z \in \menge{C}^n \setminus \{0\}$ sein EV,
		dann sind $y_1(t) = Re(e^{\lambda t} \cdot z), \quad y_2(t) = Im(e^{\lambda t} \cdot z)$ zwei linear unabh"angige reelle L"osungen von $y' = Ay$ 
		und $\overline\lambda$ kann ausgeschlossen werden.
	\item{} Eigenvektoren zu verschiedenen Eigenwerten sind linear unabh"angig.
	\end{enumerate}

\subsection{Satz von Picard-Lindel"of}
	Gegeben folgendes Anfangswertproblem: $y' = f(t,y) \quad y(t_0) = x_0$ \\
	Sei $\phi$ stetig und $\frac{\partial f}{\partial y'_j} \quad \forall j.1 \leq j \leq n$ stetig,
	dann hat AWP eine eindeutige L"osung in der lokalen Umgebung von $t_0$


Plot von $\sin x \cos x$: http://www.wolframalpha.com/input/?i=plot+sin+x+cos+y





%
%
%
%
%
%
%
%
% end!
\end{document}

