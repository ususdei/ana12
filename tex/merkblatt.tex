\documentclass[10pt,a4paper^, twocolumn]{article}
\usepackage[utf8x]{inputenc}
\usepackage{amssymb}
\usepackage{amsmath}
\usepackage{hyperref}
\usepackage[ngerman]{babel}
\usepackage[noheadfoot,top=1.5cm,bottom=2.5cm,left=1.5cm,right=1.5cm]{geometry}
\usepackage{fancyhdr}
\usepackage{wasysym} % \smiley :)
\usepackage{scalefnt}
\usepackage{graphicx} % scalebox, rotatebox


%%% AND: \wedge; OR: \vee
%%% neue befehle -- bitte benutzen ;-)

\newcommand{\basis}{\mathfrak} % basis in frakturschrift
\newcommand{\menge}{\mathbb} % menge in mengenschrift ^^
\newcommand{\und}{\wedge} % logisches und
\newcommand{\oder}{\vee} % logisches oder
\newcommand{\entspr}{\widehat{=}} % entspricht zeichen
\newcommand{\liminfty}[1]{\lim \limits_{#1 \rightarrow \infty}} % lim {#1} --> infty
\renewcommand{\epsilon}{\varepsilon} % *das* epsilon!
\renewcommand{\phi}{\varphi} % *das* phi!
\newcommand{\pivot}{{\fboxsep1pt \fbox{$\ast$}}} % pivot-element in stufen-matrix
\renewcommand{\d}{\mathrm{d}} % d/dx schreiben als \frac{\d}{\d x}
\newcommand{\R}{\mathbb{R}}
\newcommand{\N}{\mathbb{N}}
\newcommand{\C}{\mathbb{C}}
\newcommand{\I}{\mathbb{I}}


%%% neue befehle -- bitte benutzen ;-)
\makeatletter
\renewcommand*\env@matrix[1][*\c@MaxMatrixCols c]{%
  \hskip -\arraycolsep
  \let\@ifnextchar\new@ifnextchar
  \array{#1}}
\makeatother

\parindent 0mm % Absatzeinzug auf 0 setzen
\lhead{} % Überschrift
\rhead{} % seitenzahl ausblenden -- erst nach TOC anzeigen
\cfoot{} % seitenzahl ausblenden
\renewcommand{\headrulewidth}{0mm} % header trennlinie ausblenden

\begin{document}
\scalefont{0.8} % schriftgröße skalieren: 10pt --> 8pt

\begin{center}
  {\large Analysis Merkblatt}\\
  {}
\end{center}

\section{Zahlensysteme}
\subsection{Axiomatische Charakterisierung der Reellen Zahlen}
\subsection{Komplexe Zahlen}
\subsection{Einige nützliche Bezeichungen}
\subsection{Rechenregeln für Suprema}
\subsection{Archimedizität der Reellen Zahlen}
\subsection{Dichtheit der Rationalen Zahlen}
\subsection{Dezimaldarstellung}

\section{Ungleichungen}
\subsection{Elementare Ungleichungen}
Dreiecksungleichung: \\
$|a+b| \leq |a|+|b|$ \\
Umgekehrte Dreiecksungleichung: \\
$|a+b| \geq ||a|-|b||$ \\
Ungleichheit für geometrische Summen:\\
$1+x+x^2+x^3+\dots+x^n \leq \frac{1}{1-x} \quad (n \in \menge{N}_0, 0 \leq x < 1)$ \\
Bernoulli'sche Ungleichung:\\
$(1+x)^n \geq 1+nx \quad (n \in \menge{N}_0, x > -1)$ \\
Ungleichung zwischen geometrischem und arithmetischem Mittel: \\
$\sqrt{ab} \leq \frac{a+b}{2} \quad (a,b \geq 0)$ \\
Ungleichung vom mittleren Verhältnis:\\
	$
	\min(\frac{a_1}{b_1}, \dots, \frac{a_n}{b_n}) 
	\leq \frac{a_1+\dots+a_n}{b_1+\dots+b_n} 
	\leq \max(\frac{a_1}{b_1}, \dots, \frac{a_n}{b_n})
	\quad (b_1,\dots,b_n > 0)
	$
\subsection{Cauchy-Schwarz'sche Ungleichung}
$\sum\limits_{k=1}^n x_ky_k \leq \sqrt{\sum\limits_{k=1}^n x_k^2} \cdot \sqrt{\sum\limits_{k=1}^n y_k^2} \quad (x_1,\dots,x_n,y_1,\dots,y_n \in \menge{R})$ \\
oder auch: $\langle x,y \rangle \leq ||x|| \cdot ||y|| \quad (x,y \in \menge{R}^n)$
\subsection{Euklidische Norm}
$||x|| := \sqrt{\langle x,x \rangle} = \sqrt{\sum\limits_{k=1}^n x_k^2}$ \quad f"ur $x \in \menge{R}^n$
\subsubsection{Euklidisches Skalarprodukt}
$\langle x,y \rangle := \sum\limits_{k=1}^n x_k \cdot y_k$ \quad f"ur $x,y \in \menge{R}^n$ \\
andere Normen

\section{Folgen}
\subsection{Konvergenz von Folgen}
	Die Folge $(a_n)$ konvergiert gegen den Grenzwert $a \in \R$, falls für
	jede Genauigkeit $\epsilon > 0$ die Abschätzung $|a_n - a| \leq \epsilon$ für fast
	alle $n \in \N$ eingehalten wird.
\subsubsection{Monotonie der Grenzwertbildung}
	Es seien $a_n \to a$ und $b_n \to b$.\\
	Gilt $a_n \leq b_n \, \forall n \geq n_0 \in \N$, dann gilt $ a \leq b$.
\subsubsection{Einschließungsregel}
	Es sei $\forall \, n \geq n_0 \in \N : \, a_n \leq a_n' \leq a_n''$.
	Dann gilt 
	$$\lim_{n \to \infty} a_n = \lim_{n \to \infty} a_n'' = a 
	\qquad \Rightarrow  \qquad \lim_{n \to \infty} a_n' = a$$
\subsubsection{Konvergenz in $\R^d$ und $\mathbb{C}$}
	Die vektorwertige Folge\\
	$\N \ni n \mapsto x_n = 
		\begin{pmatrix}
			x_{n_1}\\
			\vdots \\
			x_{n_d}
		\end{pmatrix}$
	konvergiert gegen den Grenzwert $
		x = \begin{pmatrix}
			x_{1}\\
			\vdots \\
			x_{d}
		\end{pmatrix}$\\
	$$\Leftrightarrow 
		\lim_{n \to \infty} x_{n_k} = x_k 
		\quad \forall k \in \{1, \dots, d \} 
		\text{(komponentenweise Grenzwerte existieren)}$$
	Analog in $\mathbb{C}$ (isomorph zu $\R^2$).
\subsection{Beschränktheit konvergenter Folgen}
	Eine konvergente Folge $(a_n) \subset \R$ ist beschränkt.\\
	D.h. $\exists c > 0 $ mit $|a_n| \leq c \forall n \in \N$.\\
	Das gleiche gilt für Folgen aus  $\R^d$ und $\mathbb{C}$ mit $||\cdot||_2$
	bzw. dem Absolutbetrag in $\mathbb{C}$.
\subsection{Stetigkeit: Rechnen mit Grenzwerten}
	Eine Funktion $f : D \subseteq \R^d \to \R^q$ heißt 
	im Punkt $x \in D$ ihres Definitionsbereichs stetig, falls für jede Folge 
	$(x_n)$ aus $D$ gilt:
	$$
		x_n \overset{n \to \infty}{\to} x
		\qquad \Rightarrow \qquad
		f(x_n) \overset{n \to \infty}{\to} f(x)
	$$
	\subsubsection{Elementare stetige Funktionen}
	\begin{center}
	\newcommand{\infowidth}{2cm}
	\newcommand{\aufsichtwidth}{10cm}
	\begin{tabular}{|c|c|}
	\hline
	$D \subseteq \R$ &  $f(x) \in \R$ \\
	\hline
	$\R$ & $f(x) = c, c \in \R$\\
	$\R$ & $f(x) = |x|$\\
	$[0, \infty)$& $f(x) = \sqrt{x}$\\
	\hline
	$D \subseteq \R^2$ &  $f(x, y) \in \R$ \\
	\hline
	$\R^2$ & $f(x, y) = x+y$\\
	$\R^2$ & $f(x, y) = x \cdot y$\\
	$\R \times (\R \setminus \{0\}) $ & $f(x, y) = x \div y$\\
	\hline
	\end{tabular}
	\end{center}
	Summe, Produkt, Quotient zweier konvergenter Folgen ist konvergent. Für
	divergente/uneigentlich konvergente Folgen gilt das nicht.\\
	\\
	\begin{itemize}
	\item Polynome (in $\mathbb{C}$ und $\R$) sind stetig auf ganz $\mathbb{C}$ 
	bzw. $\R$.
	\item $\max(a,b) = \frac{a + b + |a-b|}{2}$ und $\min(a,b) = \frac{a+b - |a-b|}{2}$ 
	sind stetig auf $\R^2$, bzw. die entsprechenden Funktionen auch in 
	höherdimensionalen Räumen.
	\item Als Konsequenz sind auch Normen stetig.
	\end{itemize}
	
	
	\subsubsection{Stetigkeit der Verknüpfung von Funktionen}
		Sei $f : D \subseteq \R^d \to \R^q$, 
		$g : E \subseteq \R^q \to \R^p$ mit 
		$f(D) \subseteq E$ und \\
		$g \circ f : D \subseteq \R^d \to \R^p, 
		(g \circ f)(x) := g(f(x))$ die Verkfüpfung von $f$ mit $g$. 
		Dann gilt: Wenn $f$ stetig in $x \in D$ und $g$ stetig $f(x) \in E$\\
		$\Rightarrow g \circ f$ stetig in $x \in D$
\subsection{Monotone Folgen}
	$(a_n)_{n \in \N}$ monoton wachsend $\Leftrightarrow a_n \leq a_{n+1} 
	\, \forall n \in \N$\\
	$(a_n)_{n \in \N}$ monoton fallend $\Leftrightarrow a_n \geq a_{n+1} 
	\, \forall n \in \N$\\
	Strenge Monotonie gilt, falls statt $\leq, \geq$ echte Ungleichheit gilt.\\
	\subsubsection{Satz}
		Jede beschränkte, monotone Folge $(a_n)_{n \in \N} \subset \R$
		ist konvergent.
		Jede unbeschränkte, monotone Folge $(a_n)  \subset \R$ konvergiert 
		uneigentlich.
\subsection{Beschränkte Folgen}
	\subsubsection{Teilfolgen und Häufungswerte}
		Ist $(a_n) \subset \R^d$ eine beliebige Folge und 
		$(n_k)_{k \in \N} \subset \N $ 
		streng monoton wachsend, dann heißt $ (a_{n_k})_{k \in \N}$
		\textbf{Teilfolge} von $(a_n)$.\\
		Grenzwerte konvergenter Teilfolgen heißen \textbf{Häufungswerte}. Der maximale
		bzw. minimale Häufungswert werden als $\limsup$ bzw. $\liminf$ 
		bezeichnet und wie folgt definiert:
		$$ 
		\limsup_{n \to \infty} a_n := 
		\lim_{n \to \infty} \left( \sup_{k \geq n} a_k \right) ,
		\qquad \quad
		\liminf_{n \to \infty} a_n := 
		\lim_{n \to \infty} \left( \inf_{k \geq n} a_k \right)
		$$
	\subsubsection{Satz von Bolzano-Weierstrass}
		Jede beschränkte Folge $(a_n)_{n \in \N} \subset \R$ besitzt
		mindestens eine konvergente Teilfolge $(a_{n_k})_{k \in \N}$ (also min.
		einen Häufungswert).
	\subsubsection{Lemma Seite 6/14, VL 6}
		Sei $(a_n)$ beschränkt. Dann gilt: \\
		$$ \limsup_{n \to \infty} a_n = \liminf_{n \to \infty} a_n = a
		\qquad \Leftrightarrow \qquad
		\lim_{n \to \infty} a_n = a$$
	\subsubsection{Korollar Seite 7/14, VL 6}
		Jede beschränkte Folge $(a_n)_{n \in \N} \subset \R^d$ besitzt
		mindestens eine konvergente Teilfolge.\\
		$(a_n)$ ist genau dann konvergent, wenn sie genau einen Häufungswert besitzt.	
	\subsection{Exponentialfunktion}
		$e^x = \lim_{n \to \infty} \left(1 + \frac{x}{n} \right)^n, x \in \R$
	\subsubsection{Abschätzungen der Exponentialfunktion}
		$e^x \geq 1 + x \forall x \in \R$\\
		$e^{-x} \geq 1 - x \qquad \Rightarrow \qquad e^x \leq \frac{1}{1-x}$ für $x < 1$ 
		

\section{Reihen}
\subsection{Konvergenz von Reihen}
\subsubsection{Beispiele}
\subsection{Vergleichskriterien}
\subsubsection{Majorantenkriterium}
\subsubsection{Quotientenkriterium}
\subsubsection{Wurzelkriterium}

\section{Konsequenzen der Stetigkeit}
\subsection{Zwischenwertsatz} 
\subsection{Existenz von Maximum und Minimum}
\subsubsection{Lemma}
\subsubsection{Korollar}



\section{Ableiten}
	Def.: $\I \in \R$ Intervall, $f: \I \to \R$ \\
	$f$ heißt differenzierbar in $x_0 \in /I \\
	\Leftrightarrow \lim\limits_{x \to x_0} \frac{f(x) - f(x_0)}{x - x_0} = c \in \R \\
	\Leftrightarrow \forall (x_n)_{n \in \N} \subset \I$ mit $x_n \to x_0$ folgt, dass $\frac{f(x_n) - f(x_0)}{x_n-x_0} = c \in \R
	\Leftrightarrow \lim\limits_{n \to 0} \frac{f(x_0 + h)-f(x_0)}{h} = \in \R$ \\
	Falls der Grenzwert 
	$\Leftrightarrow \lim\limits_{x \to x_0} \frac{f(x) - f(x_0)}{x - x_0}$ existiert,
	schreiben wir $f'(x_0) = \lim\limits_{x \to x_0} \frac{f(x) - f(x_0)}{x - x_0}$
	\begin{itemize}
	\item $f$ diffbar in $x_0 \Rightarrow f$ stetig in $x_0$
	\item Kettenregel: $f,g$ Funktionen mit $f$ diffbar in $g(x_0), g$ diffbar in $x_0$
		$\Rightarrow (f \circ g)'(x_0) = f'(g(x_0))g'(x_0)$
	\item $f: \I \to \R, f$ injektiv $(f(x) = f(y) \Rightarrow x=y) \rightarrow f:I \to f(\I) bijektiv \Rightarrow f^{-1}$ existiert. \\
		Ist $f'(y) \neq 0 \Rightarrow (f^{-1})'(x) = \frac{1}{f'(f^{-1}(x))}$
	\item Arithmetikregeln:
		\begin{itemize}
		\item $(f + \lambda g)' = f' + \lambda g'$
		\item $(f \cdot g)' = (f' \cdot g) + (f \cdot g')$
		\item $(\frac{f}{g})' = \frac{f'\cdot g - f \cdot g'}{g^2}$
		\end{itemize}
	\end{itemize}


\subsection{Anwendung der Ableitungen}
	$f:[a,b] \to \I$ hat bei $x_0 \in [a,b]$ ein
	\begin{itemize}
	\item glob Max $\Leftrightarrow f(x) = f(x_0) \quad \forall x \in [a,b]$
	\item lok Max $\Leftrightarrow \exists \epsilon >0: \forall x \in (x_0-\epsilon, x_0+\epsilon) \cap [a,b]: f(x) \leq f(x_0)$
	\end{itemize}
	Notwendige Bedingung f"ur lokale Maxima: \\
	$f[a,b] \to \R ($diffbar in $x_0)$, ist $x_0 \in (a,b)$ lok Extremum $\Rightarrow f'(x_0) = 0$ \\
	Hinreichende Bedingung: \\
	$f: (a,b) \to \R \ 2 \times$ diffbar in $x_0 \in (a,b)$ mit $f'(x_0) = 0$
	\begin{itemize}
	\item $f''(x_0) < 0 \Rightarrow f$ lokales Maximum in $x_0$
	\item $f''(x_0) > 0 \Rightarrow f$ lokales Minimum in $x_0$
	\item $f''(x_0) = 0 \Rightarrow$ keine Aussage direkt möglich.
	\end{itemize}

\subsection{Regel von L'Hospital}
	$f,g: (a,b) \to \R$ diffbar, $a,b \in \R \cup \{\pm \infty\}$ \\
	Sei $x_0 \in (a,b)$ und gilt \\
	entweder $f(x), g(x) \to \infty$ f"ur $x \to x_0$  \\
	oder $f(x),g(x) \to 0$ f"ur $x \to x_0$ \\
	$ \Rightarrow \lim\limits_{x \to x_0} \frac{f(x)}{g(x)} = \lim\limits_{x \to x_0} \frac{f'(x)}{g'(x)} \in \R$

\subsection{Konvexit"at}
	Mengen: $D \in \R^n$ ist konvex $\Leftrightarrow \forall x,y \in D \ [x,y] \in D$ \\ 
	$[ x, y ] = \{\lambda x + (1- \lambda)y | \lambda \in [0,1] \}$ \\
	Funktionen: $D \subseteq \R^n$ konvex \\
	$f: D \to \R$ konvex 	$\Leftrightarrow f(\lambda x + (1-\lambda)y) \leq \lambda f(x) + (1 - \lambda ) f(y)$\\
	$f(a,b) \to \R \ 2 \times$ stetig diffbar.  $f$ konvex $\Leftrightarrow f''(x) \geq 0$

\subsection{Mehrdimensionales Ableiten}
	Def.: $U \in \menge{R}^n$ offen, \quad $f:U \to \menge{R}^m$ \\
	$f$ heißt diffbar in $x \in U \Leftrightarrow \exists$ lin. Abb. $A: U \to \menge{R}^m \quad A \in \menge{R}^{m \times n}$ mit
	$f(x+h) = f(x) + A(x)h + o(||h||) \rightarrow f(x+h) = f(x) + A(x)h$ \\
	$A$ ist durch $f$ eindeutig bestimmt aber abh"angig von $x$.\\ $A$ heißt Jakobimatrix. $A(x) := J_f(x) =: Df(x)$
	\begin{itemize}
	\item $m,n$ beliebig: $A(x) = \begin{pmatrix}
			\frac{\partial f_1}{\partial x_1} 	& \cdots 	& \frac{\partial f_1}{\partial x_n} \\
			\vdots					& \ddots	& \vdots	\\
			\frac{\partial f_m}{\partial x_1}	& \cdots	& \frac{\partial f_m}{\partial x_n}
		\end{pmatrix} = \begin{pmatrix}
			(\nabla f_1)^T \\
			\vdots \\
			(\nabla f_m)^T
		\end{pmatrix}$
	\item $m = 1, n \in \menge{N}$ \\
		$A(x) = (\nabla f(x))^T = \left ( \frac{\partial f}{\partial_1}, \cdots, \frac{\partial f}{partial_n} \right )$ \\
		Falls $f$ stetig partiall diffbar ist $\Rightarrow f$ ist diffbar. \\
		$D^2 f(x) = \begin{pmatrix}
			\frac{partial^2 f}{\partial x_1 \partial x_1} 	& \cdots	& \frac{partial^2 f}{\partial x_1 \partial x_n} \\
			\vdots						& \ddots	& \vdots	\\
			\frac{partial^2 f}{\partial x_n \partial x_1} 	& \cdots	& \frac{partial^2 f}{\partial x_n \partial x_n}
		\end{pmatrix}$ \\
		$D^2 f = H_f$ ist eine symmetrische $n \times n$-Matrix und heißt Hesse-Matrix.
	\end{itemize}




\section{Integral}
Defs.:  \dots \\
Falls $f:[a,b] \rightarrow \menge{R}$ beschr"ankt und monoton oder $f$ st"uckweise stetig, dann ist $f$ Riemann integrierbar.
\underline{Hauptsatz der Differential und Integralrechnung} \\
	Sei $f:[a,b] \rightarrow \menge{R} stetig, c \in [a,b]$ fest, dann $\quad F(x) := \int\limits_c^x f(y) \d y$
	\begin{enumerate}
	\item stetig diffbar
	\item $F'(x) = f(x) \quad \forall x \in [a,b]$
	\item $\int\limits_a^b f(x) \d x = F(b) - F(a)$
	\end{enumerate}
\subsection{Substitution}
	$g:[a,b] \rightarrow \menge{R}, F: g([a,b]) \rightarrow \menge{R}$ stetig diffbar, $F' = f$ \\
	$\int\limits_a^b f(y) \d y = \int\limits_{g^{-1}(a)}^{g^{-1}(b)} f(g(x)) g'(x) \d x$ \\
	$x = g^{-1}(y) \Leftrightarrow y=g(x) \quad \frac{\d y}{\d x} = g'(x)$
\subsection{uneigentliche Integrale}
	Sei $f:[a,b] \rightarrow \menge{R}$ integrierbar $\forall [\alpha, \beta], a \leq \alpha < \beta < b$ \\
	$ \Rightarrow \int\limits_a^b f(x) \d x = \lim\limits_{\beta \to b} \int\limits_a^\beta f(x) \d x$



\section{Potenzreihe}
	Sei $(a_n)_{n\in \menge{N}}$ Folge, 
	dann heißt $f:(-r,+r) \rightarrow \menge{R} \quad f(x) = \sum\limits_{k=0}^\infty a_kx^k$ Potenzreihe. \\
	$r > 0$ heißt Konvergenzradius, $r = \frac{1}{\limsup \sqrt[n]{|a_n|}}$. \\
	Welche Funktionen $f$ haben die Darstellung $f(x) = \sum\limits_{k=0}^\infty a_kx^k \quad \forall x \in (-r,r) \rightarrow$ \underline{Taylorentwicklung} \\
	$f: \menge{I} \rightarrow \menge{R}, \menge{I} = (a,b) \quad f (n+1)$-mal stetig diffbar, dann: \\
	$\forall x,a \in \menge{I}: (x) = \underbrace{ \sum\limits_{k=0}^{n} \frac{f^{(k)} (a)}{k!}(x-a)^k }_{\text{Taylorpolynom n-ten Grades von f um a}} + R_{n+1}(a,x)$ \\
	$R_{n+1}(a,x) = \begin{cases}
		\frac{1}{n!} \int\limits_a^x(x-t)^nf^{(n+1)}(t) \d t \\
		\frac{f^{(n+1)}(\xi)}{(n+1)!} (x-a)^{n+1} \quad \xi \in [x,a]
	\end{cases}$ \\
	falls f $\infty$-mal diffbar, $R_{n+1}(x) \rightarrow 0 (n \rightarrow \infty)
	\Rightarrow f(x) = \sum\limits_{k=0}^\infty \frac{f^{(k)}(a)}{k!} (x-a)^k$ \quad (Taylorreihe von f um a) \\
	$f:\menge{R} \rightarrow \menge{R}$ heißt analytisch in $x=0$, falls f als $f(x) = \sum\limits_{k=0}^\infty a_k x_k \quad \forall |x| < r$ darstellbar ist.
	\begin{itemize}
		\item $f \in C^\infty((-r,r))$
		\item $f$ analytisch, $g$ analytisch $\Rightarrow f \circ g$ analytisch mit
			$(f \circ g)(x) = \sum\limits_{k=0}^\infty \sum\limits_{j=0}^\infty (a_j \cdot b_{k-j})x^k \quad , |x| < min\{r_f, r_g\}$
		\item $a_k = \frac{f^{(k)}(0)}{k!}$
		\item $f$ analytisch, $f(0) \neq 0 \Rightarrow \frac{1}{f}$ analytisch
	\end{itemize}



\section{Kurven}
	\begin{itemize}
	\item{Def.:} Sei $\gamma : I \rightarrow \menge{R}^n$ eine diffbare Kurve, 
		dann heißt $\gamma'(t)$ Tangentialvektor (TV) oder Geschwindigkeitsvektor von $\gamma$ zum Zeitpunkt $t$ \\
		$||\gamma'(t)|| = \sqrt{\gamma'_1(t)^2 + \dots + \gamma'_n(t)^2}$ heißt Geschwindigkeit zum Zeitpunkt t \\
		$\frac{\gamma'(t)}{||\gamma'(t)||}$ heißt Tangentialeinheitsvektor (TEV) (falls $\gamma'(t) \neq 0$).
		$\gamma: I \rightarrow \menge{R}^n$ diffbar heißt regul"ar falls $\gamma'(t) \neq 0 \quad \forall t \in I$
	\item{Satz:} Eine außer an endlich vielen Stellen stetig diffbare Kurve $\gamma: I \rightarrow \menge{R}^n, I=[a,b]$ ist rektifizierbar und hat Länge
		$S(\gamma) = \int\limits_a^b || \gamma(t) || \d t$
	\item{Korollar:} Sei $f:[a,b] \rightarrow \menge{R}$ stetig diffbar, $\gamma:[a,b] \rightarrow \menge{R}^2$ \\
		$\gamma(x) = (x, f(x))$, 
		dann $S(f) = \int\limits_a^b\sqrt{1+f'^2(x)} \d x $
	\end{itemize}


\subsection{Parameterwechsel}
	Sei $\gamma : [a,b] \rightarrow \menge{R}^n$ Kurve und rektifizierbar. \\
	Ziel: $\tilde\lambda : [a', b'] \rightarrow [a,b]$ mit gleicher Spur wie $\gamma$, aber $DG = 1 \ \forall t \in [a', b']$ \\
	\begin{enumerate}
	\item{}  Sei $\phi:[a', b'] \rightarrow [a,b]$ stetig, bijektiv, dann heißt $\phi$ Parametertransformation von $[a,b]$ nach $[a',b']$ \\
	Definiere: $\tilde\gamma = \gamma \circ \phi, \tilde\gamma:[a', b'] \rightarrow \menge{R}^n$ mit $\tilde\gamma([a',b']) = \gamma(\phi([a',b'])) = \gamma([a,b])$ \\
	\item{} $S(\xi) = \int\limits_a^\xi ||\gamma'(t)|| \d t, s(0) = 0, s(b) = S(\gamma)$ \\
	$s'(\xi) = ||\gamma'(\xi)|| > 0$ \\
	$s [a,b] \rightarrow [0, S(\gamma)]$ und $s(t) = \int\limits_0^t ||\gamma'(\tau)|| \d \tau $ \\
	$s^{-1}[0,S(\gamma)] \rightarrow [a,b] \qquad s^{-1}$ wird $\phi$ sein. \\
	$\tilde\gamma'(\xi) = (\gamma \circ s^{-1})'(\xi) = \gamma'(s^{-1}(\xi)) \cdot (s^{-1})'(\xi) = 
	\gamma'(s^{-1}(\xi)\cdot \frac{1}{s'(s^{-1}(\xi))} = \gamma'(s^{-1}(\xi) \cdot \frac{1}{||\gamma'(s^{-1}(\xi))||}$
	\end{enumerate}




\section{Differentialgleichungen}
\subsection{Skalare gew"ohnliche DGL 1. Ordnung}
	lässt sich ein Anfangswertproblem schreiben als \\ $x' = f(t)g(x) \quad x(t_0) = x_0$  \\
	und sind ferner $f,g$ stetig und $g(x_0) \neq 0$, dann existiert eine lokale stetige Lösung $\phi(t)$ mit \\
	$\int\limits_{t_0}^t f(\tau) \d \tau = \int\limits_{x_0}^{\phi(t)} \frac{1}{g(\xi)} \d \xi$ \\
	Falls $g(x_o) = 0  \quad \Rightarrow \quad \phi(t) = x_0$

\subsection{Skalare lineare gew"ohnliche DGL $n$-ter Ordnung}
	$x^{(n)} + a_{n-1}x^{(n-1)} + \dots + a_0x = b(t)$ \\
	Alle Lösungen lassen sich schreiben als $\phi(t) = \phi_h(t) + \phi_s(t)$ \\
	Bestimmung der homogenen Lsg: \\
	$p(\lambda) = \lambda^n + a_{n-1}\lambda^{n-1} + \dots + a_0\lambda^0$  \\
	Seinen $\lambda_1,\dots,\lambda_s$ verschiedene Nullstellen des Polynoms $p(\lambda)$ mit den Vielfachheiten $r_1,\dots,r_s$,
	dann bilden folgende Funktionen eine Lösungsbasis $\langle \phi_h(t) \rangle$ des homogenen Problems: \\
	$e^{\lambda t}, te^{\lambda_jt},\dots,t^{r_j-1}e^{\lambda_jt} \quad ( 1 \leq j \leq s )$ \\
	Fall 1: Alle $\lambda_j$ reell $\quad \Rightarrow$ bereits reelle Lösungsbasis. \\
	Fall 2: $\lambda_k$ komplex $\Rightarrow \overline{\lambda_k}$ auch Nullstelle von $p$
		$\lambda_k = \alpha_k + i\beta_k$ und $\overline{\lambda_k}$ ersetzen durch: 
		$te^{\alpha_kt}cos(\beta_kt), t^ke^{\alpha_kt}sin{\beta_kt} \quad (1 \leq k \leq r_j-1)$

\subsection{Lineare gew"ohnliche DGL im $\menge{R}^n$}
	Sei $y = (y_1,\dots,y_n)^T \in \menge{R}^n$, dann $y' = Ay, A \in \menge{R}^{n\times n}$ Matrix \\
	Sind $y_1(t),\dots,y_n(t)$ linear unabh"angige reelle Fundamentallösungen von $y' = Ay$,
	so ist \\
	$y(t) = \alpha_1 y_1(t)+\dots+\alpha_ny_(t) \quad (\alpha_1,\dots,\alpha_n \in \menge{R}$ beliebig$)$ die allgemeine L"osung\\
	Berechnung der Fundamentall"osungen:
	\begin{enumerate}
	\item{} Sei $\lambda \in \menge{C}$ ein Eigenwert von $A$, $z \in \menge{C}^n \setminus \{0\}$ der zugeh"orige Eigenvektor,
		dann ist $y(t) = e^{\lambda t} \cdot z$ eine Lösung von $y' = Ay$  \\
		denn $y(t) = e^{\lambda t} \cdot z, Ax = Ae^{\lambda t}z = e^{\lambda t}Az = \lambda e^{\lambda t}z$ \\
		Existieren zum Eigenwert $\lambda \in \menge{C}$ linear unabh"angige Eigenvektoren $z_1,\dots,z_k$, so sind die Lösungen
		$y_1(t) = e^{\lambda t}z_1, \dots, y_k(t) = e^{\lambda t}z^k$ linear unabh"angig.
	\item{} Sei $\lambda \in \menge{C} \setminus \menge{R}$, d.h. $\lambda = \alpha + i\beta, \beta \neq 0 \quad z \in \menge{C}^n \setminus \{0\}$ sein EV,
		dann sind $y_1(t) = Re(e^{\lambda t} \cdot z), \quad y_2(t) = Im(e^{\lambda t} \cdot z)$ zwei linear unabh"angige reelle L"osungen von $y' = Ay$ 
		und $\overline\lambda$ kann ausgeschlossen werden.
	\item{} Eigenvektoren zu verschiedenen Eigenwerten sind linear unabh"angig.
	\end{enumerate}

\subsection{Satz von Picard-Lindel"of}
	Gegeben folgendes Anfangswertproblem: $y' = f(t,y) \quad y(t_0) = x_0$ \\
	Sei $\phi$ stetig und $\frac{\partial f}{\partial y'_j} \quad \forall j.1 \leq j \leq n$ stetig,
	dann hat AWP eine eindeutige L"osung in der lokalen Umgebung von $t_0$


Plot von $\sin x \cos x$: http://www.wolframalpha.com/input/?i=plot+sin+x+cos+y





%
%
%
%
%
%
%
%
% end!
\end{document}

