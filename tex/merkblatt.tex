\documentclass[10pt,a4paper^, twocolumn]{article}
\usepackage[utf8]{inputenc}
\usepackage{amssymb}
\usepackage{amsmath}
\usepackage{hyperref}
\usepackage[ngerman]{babel}
\usepackage[noheadfoot,top=1.5cm,bottom=2.5cm,left=1.5cm,right=1.5cm]{geometry}
\usepackage{fancyhdr}
\usepackage{wasysym} % \smiley :)
\usepackage{enumitem}
\usepackage{scalefnt}
\usepackage{sectsty}
\usepackage{graphicx} % scalebox, rotatebox


%%% AND: \wedge; OR: \vee
%%% neue befehle -- bitte benutzen ;-)

\newcommand{\basis}{\mathfrak} % basis in frakturschrift
\newcommand{\menge}{\mathbb} % menge in mengenschrift ^^
\newcommand{\und}{\wedge} % logisches und
\newcommand{\oder}{\vee} % logisches oder
\newcommand{\entspr}{\widehat{=}} % entspricht zeichen
\newcommand{\liminfty}[1]{\lim \limits_{#1 \rightarrow \infty}} % lim {#1} --> infty
\renewcommand{\epsilon}{\varepsilon} % *das* epsilon!
%\renewcommand{\phi}{\varphi} % *das* phi!
\newcommand{\pivot}{{\fboxsep1pt \fbox{$\ast$}}} % pivot-element in stufen-matrix
\renewcommand{\d}{\mathrm{d}} % d/dx schreiben als \frac{\d}{\d x}


%%% neue befehle -- bitte benutzen ;-)
\makeatletter
\renewcommand*\env@matrix[1][*\c@MaxMatrixCols c]{%
  \hskip -\arraycolsep
  \let\@ifnextchar\new@ifnextchar
  \array{#1}}
\makeatother

\parindent 0mm % Absatzeinzug auf 0 setzen
\lhead{} % Überschrift
\rhead{} % seitenzahl ausblenden -- erst nach TOC anzeigen
\cfoot{} % seitenzahl ausblenden
\renewcommand{\headrulewidth}{0mm} % header trennlinie ausblenden

\begin{document}
\scalefont{0.8} % schriftgröße skalieren: 10pt --> 8pt

\begin{center}
  {\large Analysis Merkblatt}\\
  {}
\end{center}

\section{Zahlensysteme}
\subsection{Axiomatische Charakterisierung der Reellen Zahlen}
\subsection{Komplexe Zahlen}
\subsection{Einige nützliche Bezeichungen}
\subsection{Rechenregeln für Suprema}
\subsection{Archimedizität der Reellen Zahlen}
\subsection{Dichtheit der Rationalen Zahlen}
\subsection{Dezimaldarstellung}

\section{Ungleichungen}
\subsection{Elementare Ungleichungen}
\subsection{Cauchy-Schwarzsche Ungleichung}
\subsection{Euklidische Norm}
\subsubsection{Euklidisches Skalarprodukt}
andere Normen

\section{Folgen}
\subsection{Konvergenz von Folgen}
\subsubsection{Monotonie der Grenzwertbildung}
\subsection{Beschränktheit konvergenter Folgen}
\subsection{Stetigkeit: Rechnen mit Grenzwerten}
\subsubsection{Elementare stetige Funktionen}
\subsection{Monotone Folgen}
\subsection{Beschränkte Folgen}
\subsubsection{Satz von Bolzano-Weierstrass}
\subsubsection{Lemma Seite 6/14, VL 6}
\subsubsection{Korollar Seite 7/14, VL 6}
\subsection{Exponentialfunktion}
\subsubsection{Abschätzungen der Exponentialfunktion}

\section{Reihen}
\subsection{Konvergenz von Reihen}
\subsubsection{Beispiele}
\subsection{Vergleichskriterien}
\subsubsection{Majorantenkriterium}
\subsubsection{Quotientenkriterium}
\subsubsection{•}

\section{Konsequenzen der Stetigkeit}
\subsection{Zwischenwertsatz} 
\subsection{Existenz von Maximum und Minimum}
\subsubsection{Lemma}
\subsubsection{Korollar}


Plot von $\sin x \cos x$: http://www.wolframalpha.com/input/?i=plot+sin+x+cos+y





%
%
%
%
%
%
%
%
% end!
\end{document}

