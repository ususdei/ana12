\documentclass[10pt,a4paper^, twocolumn]{article}
\usepackage[utf8]{inputenc}
\usepackage{amssymb}
\usepackage{amsmath}
\usepackage{hyperref}
\usepackage[ngerman]{babel}
\usepackage[noheadfoot,top=1.5cm,bottom=2.5cm,left=1.5cm,right=1.5cm]{geometry}
\usepackage{fancyhdr}
\usepackage{wasysym} % \smiley :)
\usepackage{scalefnt}
\usepackage{graphicx} % scalebox, rotatebox


%%% AND: \wedge; OR: \vee
%%% neue befehle -- bitte benutzen ;-)

\newcommand{\basis}{\mathfrak} % basis in frakturschrift
\newcommand{\menge}{\mathbb} % menge in mengenschrift ^^
\newcommand{\und}{\wedge} % logisches und
\newcommand{\oder}{\vee} % logisches oder
\newcommand{\entspr}{\widehat{=}} % entspricht zeichen
\newcommand{\liminfty}[1]{\lim \limits_{#1 \rightarrow \infty}} % lim {#1} --> infty
\renewcommand{\epsilon}{\varepsilon} % *das* epsilon!
\renewcommand{\phi}{\varphi} % *das* phi!
\newcommand{\pivot}{{\fboxsep1pt \fbox{$\ast$}}} % pivot-element in stufen-matrix
\renewcommand{\d}{\mathrm{d}} % d/dx schreiben als \frac{\d}{\d x}


%%% neue befehle -- bitte benutzen ;-)
\makeatletter
\renewcommand*\env@matrix[1][*\c@MaxMatrixCols c]{%
  \hskip -\arraycolsep
  \let\@ifnextchar\new@ifnextchar
  \array{#1}}
\makeatother

\parindent 0mm % Absatzeinzug auf 0 setzen
\lhead{} % Überschrift
\rhead{} % seitenzahl ausblenden -- erst nach TOC anzeigen
\cfoot{} % seitenzahl ausblenden
\renewcommand{\headrulewidth}{0mm} % header trennlinie ausblenden

\begin{document}
\scalefont{0.8} % schriftgröße skalieren: 10pt --> 8pt

\begin{center}
  {\large Analysis Merkblatt}\\
  {}
\end{center}

\section{Zahlensysteme}
\subsection{Axiomatische Charakterisierung der Reellen Zahlen}
\subsection{Komplexe Zahlen}
\subsection{Einige nützliche Bezeichungen}
\subsection{Rechenregeln für Suprema}
\subsection{Archimedizität der Reellen Zahlen}
\subsection{Dichtheit der Rationalen Zahlen}
\subsection{Dezimaldarstellung}

\section{Ungleichungen}
\subsection{Elementare Ungleichungen}
Dreiecksungleichung: $|a+b| \leq |a|+|b|$ \\
umgekehrte Dreiecksungleichung: $|a+b| \geq ||a|-|b||$ \\
Ungleichheit für geometrische Summen: $1+x+x^2+x^3+\dots+x^n \leq \frac{1}{1-x} \quad (n \in \menge{N}_0, 0 \leq x < 1)$ \\
Bernulli'sche Ungleichung: $(1+x)^n \geq 1+nx \quad (n \in \menge{N}_0, x > -1)$ \\
Ungleichung zwischen geometrischem und arithmetischem Mittel: $\sqrt{ab} \leq \frac{a+b}{2} \quad (a,b \geq 0)$ \\
Ungleichung vom mittleren Verhältnis: $min(\frac{a_1}{b_1}, \dots, \frac{a_n}{b_n}) 
					\leq \frac{a_1+\dots+a_n}{b_1+\dots+b_n} 
					\leq max(\frac{a_1}{b_1}, \dots, \frac{a_n}{b_n})
					\quad (b_1,\dots,b_n > 0)$
\subsection{Cauchy-Schwarz'sche Ungleichung}
$\sum\limits_{k=1}^n x_ky_k \leq \sqrt{\sum\limits_{k=1}^n x_n^2} \cdot \sqrt{\sum\limits_{k=1}^n y_n^2} \quad (x_1,\dots,x_n,y_1,\dots,y_n \in \menge{R})$ \\
oder auch: $\langle x,y \rangle \leq ||x|| \cdot ||y|| \quad (x,y \in \menge{R}^n)$
\subsection{Euklidische Norm}
$||x|| := \sqrt{\langle x,x \rangle} = \sqrt{\sum\limits_{k=1}^n x_n^2}$ \quad f"ur $x \in \menge{R}^n$
\subsubsection{Euklidisches Skalarprodukt}
$\langle x,y \rangle := \sum\limits_{k=1}^n x_n \cdot y_n$ \quad f"ur $x,y \in \menge{R}^n$ \\
andere Normen

\section{Folgen}
\subsection{Konvergenz von Folgen}
\subsubsection{Monotonie der Grenzwertbildung}
\subsection{Beschränktheit konvergenter Folgen}
\subsection{Stetigkeit: Rechnen mit Grenzwerten}
\subsubsection{Elementare stetige Funktionen}
\subsection{Monotone Folgen}
\subsection{Beschränkte Folgen}
\subsubsection{Satz von Bolzano-Weierstrass}
\subsubsection{Lemma Seite 6/14, VL 6}
\subsubsection{Korollar Seite 7/14, VL 6}
\subsection{Exponentialfunktion}
\subsubsection{Abschätzungen der Exponentialfunktion}

\section{Reihen}
\subsection{Konvergenz von Reihen}
\subsubsection{Beispiele}
\subsection{Vergleichskriterien}
\subsubsection{Majorantenkriterium}
\subsubsection{Quotientenkriterium}
\subsubsection{•}

\section{Konsequenzen der Stetigkeit}
\subsection{Zwischenwertsatz} 
\subsection{Existenz von Maximum und Minimum}
\subsubsection{Lemma}
\subsubsection{Korollar}

\section{Differenzialgleichungen}
\subsection{Skalare gew"ohnliche DGL 1.Ordnung}
lässt sich ein Anfangswertproblem schreiben als \\ $x' = f(t)g(x) \quad x(t_0) = x_0$  \\
und sind ferner $f,g$ stetig und $g(x_0) \neq 0$, dann existiert eine lokale stetige Lösung $\phi(t)$ mit \\
$\int\limits_{t_0}^t f(\tau) \d \tau = \int\limits_{x_0}^{\phi(t)} \frac{1}{g(\xi)} \d \xi$ \\
Falls $g(x_o) = 0  \quad \Rightarrow \quad \phi(t) = x_0$

\subsection{Skalare lineare gew"ohnliche DGL n-ter Ordnung}
$x^{(n)} + a_{n-1}x^{(n-1)} + \dots + a_0x = b(t)$ \\
Alle Lösungen lassen sich schreiben als $\phi(t) = \phi_h(t) + \phi_s(t)$ \\
Bestimmung der homogenen Lsg: \\
$p(\lambda) = \lambda^n + a_{n-1}\lambda^{n-1} + \dots + a_0\lambda^0$  \\
Seinen $\lambda_1,\dots,\lambda_s$ verschiedene Nullstellen des Polynoms $p(\lambda)$ mit den Vielfachheiten $r_1,\dots,r_s$,
dann bilden folgende Funktionen eine Lösungsbasis $\langle \phi_h(t) \rangle$ des homogenen Problems: \\
$e^{\lambda t}, te^{\lambda_jt},\dots,t^{r_j-1}e^{\lambda_jt} \quad ( 1 \leq j \leq s )$ \\
Fall 1: Alle $\lambda_j$ reell $\quad \Rightarrow$ bereits reelle Lösungsbasis. \\
Fall 2: $\lambda_k$ komplex $\Rightarrow \overline{\lambda_k}$ auch Nullstelle von $p$
	$\lambda_k = \alpha_k + i\beta_k$ und $\overline{\lambda_k}$ ersetzen durch: 
	$te^{\alpha_kt}cos(\beta_kt), t^ke^{\alpha_kt}sin{\beta_kt} \quad (1 \leq k \leq r_j-1)$

\subsection{Lineare gew"ohnliche DGL im $\menge{R}^n$}
Sei $y = (y_1,\dots,y_n)^T \in \menge{R}^n$, dann $y' = Ay, A \in \menge{R}^{n\times n}$ Matrix \\
Sind $y_1(t),\dots,y_n(t)$ linear unabh"angige reelle Fundamentallösungen von $y' = Ay$,
so ist \\
$y(t) = \alpha_1 y_1(t)+\dots+\alpha_ny_(t) \quad (\alpha_1,\dots,\alpha_n \in \menge{R}$ beliebig$)$ die allgemeine L"osung\\
Berechnung der Fundamentall"osungen:
\begin{enumerate}
\item{} Sei $\lambda \in \menge{C}$ ein Eigenwert von $A$, $z \in \menge{C}^n \setminus \{0\}$ der zugeh"orige Eigenvektor,
dann ist $y(t) = e^{\lambda t} \cdot z$ eine Lösung von $y' = Ay$  \\
denn $y(t) = e^{\lambda t} \cdot z, Ax = Ae^{\lambda t}z = e^{\lambda t}Az = \lambda e^{\lambda t}z$ \\
Existieren zum Eigenwert $\lambda \in \menge{C}$ linear unabh"angige Eigenvektoren $z_1,\dots,z_k$, so sind die Lösungen
$y_1(t) = e^{\lambda t}z_1, \dots, y_k(t) = e^{\lambda t}z^k$ linear unabh"angig.
\item{} Sei $\lambda \in \menge{C} \setminus \menge{R}$, d.h. $\lambda = \alpha + i\beta, \beta \neq 0 \quad z \in \menge{C}^n \setminus \{0\}$ sein EV,
dann sind $y_1(t) = Re(e^{\lambda t} \cdot z), \quad y_2(t) = Im(e^{\lambda t} \cdot z)$ zwei linear unabh"angige reelle L"osungen von $y' = Ay$ 
und $\overline\lambda$ kann ausgeschlossen werden.
\item{} Eigenvektoren zu verschiedenen Eigenwerten sind linear unabh"angig.
\end{enumerate}


Plot von $\sin x \cos x$: http://www.wolframalpha.com/input/?i=plot+sin+x+cos+y





%
%
%
%
%
%
%
%
% end!
\end{document}

